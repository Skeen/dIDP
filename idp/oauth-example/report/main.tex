\documentclass[12pt, a4paper]{article}
\usepackage{mystyle}

\begin{document}

\title{
\small Identity and Privacy \\
Course project II \\
\huge \textsc{DIY OAuth 2.0}
}

\author{Emil 'Skeen' Madsen - 20105376 \\
\href{mailto:sverre@cs.au.dk}{Sverre Bisgaard Rasmussen - 20083549}
}

\date{}

\maketitle

\hrule

\section*{Introduction}

  In this project we will try to showcase a naive implementation of
  the
  \href{http://tools.ietf.org/html/rfc6749}{OAuth 2.0 Framework}.
  We will dig a little into the threat model of OAuth 2.0 and look at
  some security aspects of our implementation.
  Finally, we will draw a comparison to the OpenID Connect protocol.

  It should be noted that this particular implementation are not
  production quality and should not be
  used in practice.
  It would be a better idea to use a widely distributed and acknowledged
  library for an OAuth implementation.

\section*{Naive Implementation of OAuth 2.0}

  We have implemented the 'Implicit Grant' model, which is good for
  public clients with a fixed address.
  Also we have taken a lot of shortcuts so fx, is the client registration
  phase faked.
  See the Disclaimer section for more such shortcuts.

  We have three servers running:
  \begin{description}
    \item[Client] Python based webserver
      serving static files. \\
      \url{http://mozart.cs.au.dk:45451}
    \item[Resource server]
      Node.js based. \\
      \url{http://mozart.cs.au.dk:45454}
    \item[Authentication
      server]
      Node.js based. \\
      \url{http://mozart.cs.au.dk:45455}
  \end{description}

  Notice, the web servers are only available from the university network
  (and only as long as the machines are not rebooted).
  It should be possible to access this network, either from the physical
  network or via VPN or (ssh) proxy.
  To test the servers, we recommend tuning a browser to the client
  \url{http://mozart.cs.au.dk:45451}.


  \subsection*{Client}
  This can essentially be replaced by any conforming webserver, we've
  decided to go with a python webserver because it's simple (12 lines of
  code) and easy to run.

  \subsection*{Authorization and Resource servers}

  We considered several options for the two other server entities,
  everything from Java Servlets and Dart to Python.
  We did a bit of research on how to implement HTTP servers in Dart, but
  decided to go with a language, that we have some experience with:
  JavaScript on a Node.js server.

\section*{Disclaimer}
  We've made several assumptions to simplify the implementation of the
  system.
  - For instance to avoid having a database at each service entity.

  These are obvious shortcomings of our implementation, but does not
  interfere with the main objective of the system.
  Any real implementation should avoid these simplifications, as they
  severely limit the usability of the system.

  \begin{itemize}
    \item
      At the index.html on the client (landing page), we assume that the
      user is not logged in. \\
      The correct thing to do, would be to check if the user is indeed
      logged in, and generate content based on that (for instance,
      hiding the login button, or replacing it with a logout
      equivalent).

    \item
      We currently only support a single clientID, namely the hard-coded
      'brain' clientID. \\
      The correct thing to do, would be for the auth-server to have
      database of valid clientIDs, instead of using hardcoded entities.

    \item
      We've decided to leave the scope parameter, as our application
      contains only a single scope. \\
      In order to support multiple scopes, this parameter should be
      passed around and registered with the access token, such that the
      access token would only yield access to the given scope.

    \item
      At the login.html on the auth server (login and accept scope
      page), we assume that the user has been confronted with a login
      prompt and logged in already. That is, we have faked
      authentication for the user on the authorization server.
      Hence we simply welcome the user (user1342). \\
      The correct thing to do would be to actually authenticate the user
      at the server. Fx through a username and password.

    \item
      At the login.html on the auth server, the user is not able to
      decline accepting the scope.
      Trying to do so, will simply disable the 'cancel' button.

    \item
      We've currently hard-coded the access token to '1337'.
      This means there's only a single access token for the entire
      system. \\
      The correct thing to do would be to actually generate the access
      tokens when requested.
      The auth-server would then connect to the resource-server and
      register it, or they'd simply share the same underlying database
      server of access tokens.

    \item
      The returnURLs are currently hard-coded.  Which means the user
      will be returned to a specific site, rather then the one they came
      from.
      - This is not an issue in our system, due to the limited number of
      pages. \\
      The correct thing to do would be to actually pass the returnURL as
      a parameter, and return to it when a return is indeed needed.

    \item
      The client server doesn't itself pull data from the server.
      This is done from the javascript of the served page.
      While this is strictly speaking incorrect, it doesn't really
      matter, as the client server is indeed passed the access token
      (i.e.  it is in the server log).
      - We've made a button on the loggedin.html page, which emulates
      the server actually asking the resource server for data. \\
      The correct thing to do would be to actually pull the data from
      the client server, which would be necessary, if the client for
      instance wanted to build their own database with the resource
      servers provided data.
      - Or if the client was to serve static data to the user, rather
      than have the users javascript built up the browser page.

  \end{itemize}

\section*{Threat Model for OAuth 2.0}

  \subsection*{Threat: Optaining Access Tokens}

  An adversary listening to the connection between the
  authorization server and the user (resource owner)
  may obtain the access token, because the access token is
  transmitted in the URI fragment.

  The solution is to use TLS in the communication between the user
  and the authz server.
  This is actually also specified in the
  \href{http://tools.ietf.org/html/rfc6749}{OAuth 2.0
  specification}.

  \subsection*{Threat: Guessing the credentials}

  An adversary may attempt to authenticate as the resource owner against
  the authorization server.

  In our implementation this is guaranteed to succeed as the
  authentication process is faked.

\section*{Comparison with OpenID Connect protocol}

  The OpenID Connect protocol builds on top of OAuth 2.0.
  It emphasizes the authentication of the resource owner and generally
  specifies the parameters exchanged in the protocol.

  \subsection*{Changes to the client registration phase}

  Now the client registration should be made dynamically and initiated
  on a discovery mechanism (that the client instigate).

  \subsection*{Changes to the Implicit}

  Regarding authentication of the resource owner, the parameters are
  made very specific.
  We should explicitly add parameters for scope, return pointer and a
  nonce.

  \section*{Conclusion}

  Our implementation definitely suffers from the prototype status and
  lots of faked code regarding user authentication and client
  registration.

  Looking at OpenID Connect, these things are much more specifically
  mentioned how to implement, and as such it feels safer to use.

\end{document}
