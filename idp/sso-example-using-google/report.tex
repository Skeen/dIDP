\documentclass[12pt, a4paper]{article}

\usepackage{mystyle}

\begin{document}

\title{\small Identity and Privacy\\
       Course Project I\\
       \huge \textsc{SSO Example using Google Plus}\\
       }
       \author{
        \href{mailto:sverre@cs.au.dk}{Sverre Bisgaard Rasmussen - 20083549}\\
       }

\maketitle
\hrule

\subsection*{What I did}
I chose Google for this project. I did not want to create a facebook.com
profile, which could have been an alternative.

Google supplies succinct documentation, and I found a nice tutorial on
th subject, which I followed:
\url{https://developers.google.com/+/quickstart/javascript}

As such a website were created with a nice "Log in with Google Plus"
button.

\subsubsection*{Web page:}
\begin{comment}
\url{http://localhost:45414}, \url{http://birc02.daimi.au.dk:45414}

\subsubsection*{Client Id:}
\url{405157659036-hd07ho25rj6bhen0v5taq7pod4ifm25f.apps.googleusercontent.com}
\end{comment}

Server is running from \url{http://Birc02.daimi.au.dk:45414}.

Notice, this is available ONLY from the university network. Use a vpn or
proxy (fx through ssh) to access Birc02 like the normal llama workhorse.
If this is not possible, please just download the file from the network
({\tt 'curl -O http://Birc02.daimi.au.dk:45414/index.html'}) and serve
it yourself from localhost with fx ‘{\tt python -m SimpleHTTPServer}’.

\subsubsection*{Result}

This created a server pulling my information from google plus when I
accessed it.

\subsection*{Playing around with the API}

The simple test website were customized in the following way:
\begin{itemize}
\item A header line were added
\item A new section were added at the top
\item A new script were inserted to append information to the section
  mentioned above. It uses the API to fetch:
  \begin{itemize}
  \item Age range of the ‘me’ user (see
    \url{https://developers.google.com/+/api/latest/people#resource})
  \item Relationship status of the ‘me’ user \\
    If the relationship status of the ‘me’ user is ‘single’ then a list of
    friends are shown.
  \end{itemize}
\end{itemize}

All changes are clearly marked in the code.

\subsection*{Flow}

The website accesses the Identity of the user through Google for
authentication.

The following is a list of attributes accessed:
\begin{itemize}
  \item Age range
  \item Display name
  \item List of circled people (As strictly allowed by the user in
    contact with google)
\end{itemize}

The access is granted in the following way:

\begin{itemize}
  \item The developer sets up an account at google and requests a
    project at their ‘Developer console’.
    The developer console is where the developer:
    \begin{itemize}
      \item puts requests for use of Google API’s.
        This project requests the ‘Google+ API’
      \item requests credentials from Google.
        This comes in the form of (OAuth 2.0) a client ID and a client
        secret.
      \item setup javascript origin which is so google can provide
        authentication to only the sites that the developer specifies.
    \end{itemize}
  \item The web app uses JavaScript to obtain authenticated user info
    (classical identity provider)
    \begin{itemize}
      \item User initiates access request by pressing a button on the webpage
      \item The button fires some JavaScript (from google’s API) with the following
        custom info:
        \begin{itemize}
          \item Scope: ‘plus-login’
          \item client ID
          \item callback
            \begin{itemize}
              \item where to enter the JavaScript when the request is answered by google.
            \end{itemize}
        \end{itemize}
      \item The user signs in to google and verifies what to disclose
        (which circled people to disclose).
      \item The web-app (the users JavaScript klient) receives a signed
        token from google in the callback
        (‘{\tt authResult['access\_token']}'), which is used to access
        information through Google.
      \item The token is fx used to access the display name and age
        range of the user via a Google API call:

        {\tt gapi.client.plus.people.get({ 'userId': ‘me' })}

      \item If the page is not automatically refreshed (or anything went
        wrong), the web app presents the user with personal information
        using calls to Google via the signed token.
      \item The signed token can also be used outside of JS-API calls.
        Google provides a HTTP API so the following works:
        \begin{itemize}
          \item Read and copy the 'access-token' variable from the
            signed token from google.
          \item Retrieve the userID.
          \item Replace YYY with the userID and XXX with the value of
            the access-token in the following line:

            {\tt curl -X GET
              '\url{https://www.googleapis.com/plus/v1/people/YYY?access\_token=XXX}}
        \end{itemize}
    \end{itemize}
\end{itemize}

\subsection*{OAuth Notes}
The flow above about how the access is granted matches very well with
the OAuth 2 protocol (as described in the PDF from Alexandra (See
section 2.2 of "Comparative analysis - Web-based Identity Management
Systems”).

The flow seems to be of the type 'Implicit Grant' – which is optimized
for public clients known to operate a particular redirection URI.

\subsubsection*{Client Registration}
The web app (client) has been registered at Google with the information
about client type (‘public') and the client Redirection Endpoint (yet to
be tested — google (the Authz Server) should publish the authorization
credentials here).

Less important are the shared information about:

\begin{quote}the application name, description, logo image, acceptance
  of legal terms, etc…\end{quote}

A unique client ID is agreed upon.

A client secret were also exchanged

\subsubsection*{Resource owner access grant - Implicit grant}
\begin{description}
  \item[(1., 2., 3.)] The user (Resource Owner) initiates contact and is served a
JavaScript that redirects to Google (on button clicked)
\item[(4., 5., 6., 7.)] The resource owner agrees with Google (Authorization
server) about the scope
\item[(8., 9., 10.)] The authorization server sends an encryption of an access
token to the resource owner and redirects to the client’s web app (I
might have made a mistake, because sometimes I have to refresh the
webpage to get it to work).
\item[(11., 12., 13.)] The client's web app gets the access token (It is not
send to the Client)
\item[(14.)] The resource owner’s browser should send the token to the client.
I have not implemented this.
\end{description}

\subsubsection*{Client Access}
If the last point (14.) had been implemented, the client data could be
accessed like this:
\begin{quote}{\tt curl -X GET
    '\url{https://www.googleapis.com/plus/v1/people/YYY?access\_token=XXX'}}\end{quote}

\subsection*{Goals Achieved}
\subsubsection*{Selective disclosure}
The user is able to select level of disclosure regarding the list of
circled people.
\subsubsection*{Minimal disclosure}
The user only discloses an unspecified age in the attribute: ‘ageRange’
\subsubsection*{Anonymity (Untraceability)}
Google knows everything
\subsubsection*{Pseudonymity (Unlinkability)}
It is not obvious, if Google allow multiple Google+ profiles. If that is
the case, it might provide pseudonymity.

% that's all folks
\end{document}
